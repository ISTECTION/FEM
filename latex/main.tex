\documentclass[12pt,a4paper]{article}
\usepackage[left=20mm,top=20mm,total={170mm,257mm}]{geometry}

\usepackage{styles/preamble}
\usepackage{minted}                 % Для вставок с кодом
\usepackage{xcolor}
\usepackage{mathtools}              % Для математических формул

\setmainfont[Scale=1.0]{Noto Sans}
\definecolor{LightGray}{gray}{0.9}
\usemintedstyle{vs}

% =================================================
% НАЧАЛО ДОКУМЕНТА
% =================================================

\begin{document}
    \import{titlepage/}{title}

    \section{Условие задачи}                % Условие задачи
    \subsection*{Формулировка задачи}       % Формулировка задачи

    МКЭ для двумерной краевой задачи для эллиптического уравнения в декартовой
    системе координат. Базисные функции линейные на треугольниках. Краевые условия
    всех ти-пов. Коэффициент   разложить по линейным базисным функциям.
    Матрицу СЛАУ генери-ровать в разреженном строчном формате. Для решения
    СЛАУ использовать МСГ или ЛОС с неполной факторизацией.

    \subsection*{Постановка задачи}

    Эллиптическая краевая задача для функции \textit{u} определяется дифференциальным
    уравнением

    \[ -div( \lambda grad u) + \gamma u = f \]

    \noindent заданным в некоторой области $\Omega$ с границей
    $S=S_1 \cup S_2 \cup S_3$ и краевыми условиями

    \section{Текст программы}

    \section{Тестирование}

    \section{Выводы}






\begin{minted}[linenos]{c++}
#include "argparse/argparse.hpp"
#include "timer/cxxtimer.hpp"
#include "LOS/LOS.hpp"
#include "FEM.hpp"

int main(int argc, char* argv[]) {
    using namespace ::Log;
    using ::std::chrono::milliseconds;

    argparse::ArgumentParser program("FEM", "1.0.0");
    program.add_argument("-i", "--input" ).required().help("path to input files" );
    program.add_argument("-o", "--output").required().help("path to output files");

    try {
        program.parse_args(argc, argv);

        cxxtimer::Timer timer(true);
        FEM fem      (program.get<std::string>("-i"));
        fem.writeFile(program.get<std::string>("-o"), 1E-14, 10000);
        LOS<double> l(program.get<std::string>("-o"));
        l.solve(Cond::HOLLESKY, true);
        timer.stop();

        l.printX();

        std::cout << '\n' << "Milliseconds: "
                << timer.count<milliseconds>() << '\n';
        fem.printAll();
        fem.printSparse();
    }
    catch(const std::runtime_error& err) {
        Logger::append(getLog("argc != 3 (FEM -i input -o output)"));
        std::cerr << err.what();
        std::cerr << program;
        std::exit(1);
    }
    return 0;
}
\end{minted}

\end{document}