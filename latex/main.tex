\documentclass[12pt,a4paper]{article}
\usepackage[left=20mm,top=20mm,total={170mm,257mm}]{geometry}

\usepackage{styles/preamble}
\usepackage{minted}                 % Для вставок с кодом
\usepackage{xcolor}
\usepackage{mathtools}              % Для математических формул
\usepackage{amsmath}

\setmainfont[Scale=1.0]{Noto Sans}
\definecolor{LightGray}{gray}{0.9}  % Не используется
\usemintedstyle{vs}                 % Стиль кода пакета minted


% =================================================
% НАЧАЛО ДОКУМЕНТА
% =================================================

\begin{document}
\import{titlepage/}{title}

\section{Условие задачи}                % Условие задачи
\subsection*{Формулировка задачи}       % Формулировка задачи

МКЭ для двумерной краевой задачи для эллиптического уравнения в декартовой
системе координат. Базисные функции линейные на треугольниках. Краевые условия
всех ти-пов. Коэффициент   разложить по линейным базисным функциям.
Матрицу СЛАУ генери-ровать в разреженном строчном формате. Для решения
СЛАУ использовать МСГ или ЛОС с неполной факторизацией.

\subsection*{Постановка задачи}

Эллиптическая краевая задача для функции \textit{u} определяется дифференциальным
уравнением

\[ -div( \lambda grad u) + \gamma u = f \]

\noindent заданным в некоторой области $\Omega$ с границей
$S=S_1 \cup S_2 \cup S_3$ и краевыми условиями:

\[ u \vert_{S_1} = u_g \]
\[ \lambda \frac{\partial u}{\partial n} \bigg\vert_{S_2} = \theta \]
\[ \lambda \frac{\partial u}{\partial n} \bigg\vert_{S_3}
    + \beta(u \vert_{S_3} - u_{\beta}) = 0 \]

\noindent В декартовой системе координат {x,y} это уравнение может быть записано
в виде

\[ -\frac{\partial}{\partial x}
    \left( \lambda \frac{\partial u}{\partial x} \right)
    -\frac{\partial}{\partial y}
    \left( \lambda \frac{\partial u}{\partial y} \right)
    + \gamma u = f \]

\subsection*{Конечноэлементная дискретизация}

Так как для решения задачи используются линейные базисные
функции, то на каждом конечном элементе $\Omega_k$ -
треугольнике эти функции будут совпадать с функциями
$L_1(x,y), L_2(x,y), L_3(x,y)$, такими, что $L_1(x,y)$
равна единице в вершине $(x_1,y_1)$ и нулю во всех остальных
вершинах, $L_2(x,y)$ равна единице в вершине $(x_2,y_2)$
и нулю во всех остальных вершинах, $L_3(x,y)$ равна единице
в вершине $(x_3,y_3)$ и нулю во всех остальных вершинах.
Любая линейная на $\Omega_k$ функция представима в виде
линейной комбинации этих базисных линейных функций,
коэффициентами будут значения функции в каждой из вершин
треугольника $\Omega_k$. Таким образом, на каждом конечном
элементе нам понадобятся три узла – вершины треугольника.

\[ \psi_1 = L_1(x,y) \]
\[ \psi_2 = L_2(x,y) \]
\[ \psi_3 = L_3(x,y) \]
\newpage

\noindent Учитывая построение \textit{L-функций},
получаем следующие соотношения:

\begin{equation*}
    \begin{cases}
        L_1 + L_2 + L_3 = 1          \\
        L_1x_1 + L_2x_2 + L_3x_3 = x \\
        L_1y_1 + L_2y_2 + L_3y_3 = y
    \end{cases}
\end{equation*}

\noindent Т.e. имеем систему:
\renewcommand{\arraystretch}{1.25}
\begin{equation*}
    \begin{pmatrix}
        1   & 1   & 1   \\
        x_1 & x_2 & x_3 \\
        y_1 & y_2 & y_3
    \end{pmatrix}
    \cdot
    \begin{pmatrix}
        L_1 \\
        L_2 \\
        L_3
    \end{pmatrix}
    =
    \begin{pmatrix}
        1 \\
        x \\
        y
    \end{pmatrix}
\end{equation*}
\renewcommand{\arraystretch}{1.0}

\noindent Отсюда находим коэффициенты
линейных функций $L_1(x,y), L_2(x,y), L_3(x,y)$
\[ L_i = a_0^i + a_1^ix + a_2^iy, i = \overline{1,3} \]

\renewcommand{\arraystretch}{1.25}
\begin{equation*}
    \begin{pmatrix}
        \alpha_0^1 & \alpha_1^1 & \alpha_2^1 \\
        \alpha_0^2 & \alpha_1^2 & \alpha_2^2 \\
        \alpha_0^3 & \alpha_1^3 & \alpha_2^3
    \end{pmatrix}
    =
    D^{-1}
    =
    {\begin{pmatrix}
        1   & 1   & 1   \\
        x_1 & x_2 & x_3 \\
        y_1 & y_2 & y_3
    \end{pmatrix}}^{-1}
\end{equation*}
\renewcommand{\arraystretch}{1.0}























\section{Текст программы}

\section{Тестирование}

\section{Выводы}

\begin{minted}[linenos,mathescape=true,baselinestretch=1.0]{c++}
#include "argparse/argparse.hpp"        /// $\frac{a+b}{2}$
#include "timer/cxxtimer.hpp"           /// $\frac{a+b}{2}$
#include "LOS/LOS.hpp"                  /// $\frac{a+b}{2}$
#include "FEM.hpp"

int main(int argc, char* argv[]) {
    using namespace ::Log;
    using ::std::chrono::milliseconds;

    argparse::ArgumentParser program("FEM", "1.0.0");
    program.add_argument("-i", "--input" ).required().help("path to input files" );
    program.add_argument("-o", "--output").required().help("path to output files");

    try {
        program.parse_args(argc, argv);

        cxxtimer::Timer timer(true);
        FEM fem      (program.get<std::string>("-i"));
        fem.writeFile(program.get<std::string>("-o"), 1E-14, 10000);
        LOS<double> l(program.get<std::string>("-o"));
        l.solve(Cond::HOLLESKY, true);
        timer.stop();

        l.printX();

        std::cout << '\n' << "Milliseconds: "
                << timer.count<milliseconds>() << '\n';
        fem.printAll();
        fem.printSparse();
    }
    catch(const std::runtime_error& err) {
        Logger::append(getLog("argc != 3 (FEM -i input -o output)"));
        std::cerr << err.what();
        std::cerr << program;
        std::exit(1);
    }
    return 0;
}
\end{minted}
\end{document}